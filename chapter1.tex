
\chapter{Introducción}

En la actualidad el uso de combustibles f\'osiles ha sido primordial para el desarrollo de la sociedad. En M\'exico, según datos de la Secretaria de Energ\'ia, el consumo total de energ\'ia nacional de gas natural corresponde al 49.11\% el cual mayormente es usado para las termoeléctricas de ciclo combinado y combusti\'on interna\cite{BNE2021}. Para reducir la contaminaci\'on y recuperar calor residual de cualquier proceso debido a la combustión de combustibles fósiles es de vital importancia el estudio de recuperadores de energ\'ia t\'ermica o regeneradores. En un proceso industrial t\'ermico la temperatura residual esta en un rango de $100-200 ^\cdot C$\cite{Anish2021}. 
\newline
Las aplicaciones de los intercambiadores de calor que funcionan como regeneradores de energ\'ia t\'ermica se han investigado de manera puntual según su aplicaci\'on y uso. Se pueden encontrar en la literatura, calentadores de aire solar que son muy costosos para su fabricaci\'on dependiendo del material de construcci\'on, por lo que se buscan técnicas para reducir sus costes de fabricaci\'on y que sean mas accesibles%\cite{Abbas2022}.
\newline
Estos sistemas pueden ser dise\~nados a partir de correlaciones como por ejemplo el uso del n\'umero de Nusselt, esta cantidad adimensional indica como es la transferencia de calor por convecci\'on%\cite{Pimsarn2022}, ademas el numero de Raynolds es de vital importancia para saber el r\'egimen del fluido que se esta estudiando es esta en la categoría de laminar o turbulento %\cite{Kong2022}.
\newline
Otra manera de abordar problemas t\'ermicos es mediante las ecuaciones gobernantes de conservaci\'on de masa%\cite{Zhao2023}, momentum y energ\'ia. A diferencia de la t\'ecnica por correlaciones, las ecuaciones gobernantes se pueden expresar en tres dimensiones ($x, y,$ y $z$)%\cite{Rehman2022}.
Con la ayuda de la tecnol\'ogia, se pueden simular estos sistemas para comprobar los modelos obtenidos y asi tener una referencia mas fiable. La din\'amica de fluidos computacional, es una gran herramienta que ademas de proveer informaci\'on del comportamiento del flujo en la cavidades, principalmente para los que son en medios porosos%\cite{Prajapati2022}.
\newline
Recientemente el modelado computacional y dise\~no artifical de medios porosos ha aumentado en los ultimos a\~nos debido a su aplicaci\'on en las \'areas como la biom\'edica y el\'ectrica. En la investicaci\'on de esos sistemas se enfocan en como es la distribución de los poros o cavidades, y como su geometría afecta en la forma en que se transfiere calor o en la que mejor se aprovecha%\cite{Prajapati2022}.
\newline

\subsection{Objetivo general}
Diseñar, construir, y poner en operaci\'on una planta piloto de un regenerador de energ\'ia de lecho empacado, que sirva como estaci\'on de prueba para validar estructuras matem\'aticas que representen la din\'amica de la
planta y que sirvan para la soluci\'on de problemas de diseño, optimizaci\'on y control de estos sistemas.
\subsection{Objetivos espec\'ificos}
1.- Obtener un prototipo de regenerador de energ\'ia con doble lecho empacado con un monolito met\'alico.
\newline 2.- Generar tres modelos matem\'aticos mediante diferentes m\'etodos de modelado para reproducir el
comportamiento t\'ermico de un periodo de calentamiento o enfriamiento, todos validados con datos
provenientes del prototipo experimental.
\newline 3.- Formular un modelo matem\'atico del comportamiento t\'ermico de un ciclo completo, en r\'egimen pseudo-estacionario.
\subsection{Hip\'otesis}
%\begin{itemize}
%	\item El sistema completo tiene tiempo de enfriamiento y calentamiento de ambos regeneradores es simultaneo cuando hace el cambio a contraflujo.
%	\item El aire usado para el calentamiento o enfriamiento es seco.
%	\item El sistema de suministro de aire solo funciona en las condiciones y lugar del CENIDET
%	\item Las ecuaciones dise\~{n}o del intercambiador de calor van a servir de punto de partida para probar otras t\'ecnicas de modelado.
%\end{itemize}

