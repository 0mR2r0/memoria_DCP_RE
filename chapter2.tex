% From mitthesis package
% Version: 1.06, 2024/07/09
% Documentation: https://ctan.org/pkg/mitthesis


\chapter{Capitulo}

subsubsection{C\'alculo de la velocidad de salida del compresor}
De lo anterior, la informaci\'on del caudal proveniente del compresor se encuentra indicada en pies c\'ubicos por minuto estándar, para obtener el dato de la velocidad se necesita realizar una conversi\'on a pies cúbicos por minuto (CFM)
\newline
Si los SCFM es un flujo volum\'etrico a condiciones estándar de 14.7 psi a $20^\circ C$, para convertir a CFM implica las condiciones actuales de operaci\'on, de tal manera que se deben tomar las condiciones de presi\'on atmosférica y temperatura de la ubicación geográfica. Para este caso se tomaran las condiciones del CENIDET, con los siguientes parámetros:
\begin{itemize}
	\item Presión atmosférica: 0.9938317 atm
	\item Temperatura ambiental normal: $23.2^\circ C$
\end{itemize}
Según la Ley de los gases ideales, la densidad es proporcional a la presión e inversamente proporcional a la temperatura\cite{Cengel2022}:
\begin{equation}
	PV=mRT
\end{equation}
\begin{equation}
	\rho=\frac{m}{V}=\frac{P}{RT}
\end{equation}
Donde:
\newline
$P$ es la presión expresado en psi(pulgadas por centímetro cuadrado).
\newline
$V$ es el volumen expresado en $m^3$.
\newline
$m$ es la masa del gas expresado en kilogramos.
\newline
$R$ es la constante de los gases ideales.
\newline
$T$ es la temperatura expresada en grados Rankine ($^\circ R$).
\newline
$\rho$ es la densidad del aire expresada en $\frac{Kg}{m^3}$
\newline
Por lo tanto se puede deducir la siguiente formula para convertir de SCFM a CFM:
\begin{equation}
	CFM=\frac{SCFM}{\frac{P_{actual}}{14.7}*\frac{528^\circ R}{T_{actual}}}
\end{equation}
Realizando las respectivas sustituciones y convirtiendo el valor de la temperatura de Celsius a  Rankine, y la presi\'on de atmósferas a psi se tiene:
\begin{equation}
	CFM=\frac{3.7}{\frac{14.62}{14.7}*\frac{528^\circ R}{533.43^\circ R}}
\end{equation}
Por lo tanto los CFM que proporciona el compresor seleccionado es de 3.7585 CFM.
\newline
Para tener el valor correspondiente de la velocidad del flujo de aire después de la electrov\'alvula, se necesita realizar la conversión de CFM a $\frac{l}{min}$. Donde 1 CFM es equivalente a 0.47 $\frac{l}{s}$. De la capacidad de flujo obtenido anteriormente el caudal que puede proporcionar es: 1.7738 $\frac{l}{s}$
\newline
El caudal viene definido por la siguiente expresi\'on donde:
\begin{equation}
	Q=\frac{V}{t}
\end{equation}
para conocer la velocidad de flujo de un caudal dado se esta considerando una tubería, se considera el área transversal de la tubería($A=\pi r^2$) de selección para un diámetro de 5mm de diámetro. 
\newline
La ecuación anterior queda de la siguiente manera: 
\begin{equation}
	v=\frac{Q}{\pi r^2}
\end{equation}
Por lo tanto la velocidad que puede proporcionar el compresor a través de la electrov\'alvula para un diámetro de 5 mm es aproximadamente de: \textit {90.339} $\frac{m}{s}$\cite{Atlas2014}