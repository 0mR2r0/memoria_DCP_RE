
\chapter{Marco conceptual}

\section{Almacenamiento de energ\'ia t\'ermica}
En el desarrollo de sistemas de almacenamiento de energ\'ia t\'ermica se deben considerar los siguientes factores:
\begin{itemize}
	\item \textbf{Capacidad}: Esta característica del sistema depende el proceso de almacenamiento, el medio y el tama\~no que tendrá, por ejemplo el agua tiene una capacidad de almacenar calor de $10-50 \frac{kWh}{t}$.La capacidad de almacenamiento tiene un importante efecto en la operación del resto del sistema.
	\item \textbf{Potencia}: Es la raz\'on de carga y descarga relacionados con la potencia necesaria. De manera general, se define que tan r\'apido la energ\'ia puede almacenarse en un sistema y descargarse.
	\item \textbf{Eficiencia}: Esta relaci\'on entre la energ\'ia que suministra el usuario con la energ\'ia que se requiere para cargar el sistema. Esto considera la perdida de energ\'ia durante el almacenamiento en un periodo de carga y descarga. Generalmente la eficiencia del calor sensible en un sistema esta en un rango del 50$\%$ - 90$\%$
	\item \textbf{Periodo de almacenamiento}: La evaluaci\'on económica de un sistema de almacenamiento t\'ermico depende significativamente de la aplicaci\'on, los requerimientos de operaci\'on y la frecuencia de almacenamiento.
	\item \textbf{Costo}: Esto se refiere a la capacidad o potencia del almacenamiento del sistema relacionado con los costos de operaci\'on, equipo de almacenamiento y tiempo de vida \'util. El costo incluye desde el inicio del almacenamiento medio, contenedores y aislamientos. De manera general, los costos de estos sistemas pueden variar entre 0.15-10$\frac{\$}{kWh}$ para sistemas de almacenamiento de calor sensible.
\end{itemize}
Cabe mencionar que una buena unidad de almacenamiento debe tener una eficiencia mayor al 80$\%$. Los requerimientos t\'ecnicos para estos sistemas deben incluir:
\begin{itemize}
	\item Alta densidad energ\'etica en el material del almacenamiento, que se traduce en el costo por el espacio ocupado y su aislamiento.
	\item Buena transferencia de calor entre el fluido de transferencia y el medio de almacenamiento con baja p\'erdida t\'ermica.
	\item Mec\'anica y qu\'imicamente estable material de almacenamiento, entre mejor sea su costo incrementa.
	\item Reversibilidad para gran cantidad ciclos de carga y descarga. El costo del intercambiador para la carga y para la descarga es importante\cite{Yatish2017}.
\end{itemize}

\subsection{Almacenamiento de calor sensible}
Los almacenes de calor sensible, usan la capacidad calor\'ifica y el cambio de la temperatura del material del almacén durante el proceso de carga y descarga. La cantidad de calor almacenado depende de su capacidad calor\'ifica, de su cambio de temperatura y la cantidad de almacenamiento t\'ermico que posee el material de interés\cite{Yatish2017}.
\begin{equation}
	Q=\rho V C_p \Delta T
\end{equation}
Donde: 
\newline
$Q$ es la cantidad de calor almacenado en Joules [$J$] 
\newline
$\rho$ es la densidad del material de almacenamiento [$\frac{kg}{L}$]
\newline
$C_p$ es el calor espec\'ifico sobre el rango de temperatura de operaci\'on [$\frac{J}{kg K}$]
\newline
$V$ es el volumen del material de almacenamiento usado [$L$]
\newline
$\Delta T$ es el rango de la temperatura operaci\'on [$^\circ C$]
\newline
\newline
La conductividad  t\'ermica del material afecta directamente la raz\'on carga y la descarga  en el almacenamiento, y se puede obtener de la siguiente expresi\'on:
\begin{equation}
	\lambda=\rho C_p \alpha
\end{equation}
Donde:
\newline
$\lambda$ es la conductividad t\'ermica [$\frac{W}{m K}$]
\newline
$\rho$ es la densidad [$\frac{kg}{m^3}$]
\newline
$C_p$ es la capacidad calor\'ifica [$\frac{J}{kg K}$]
\newline
$\alpha$ es la difusividad t\'ermica [$\frac{m^2}{s}$]