
\chapter{Marco conceptual}

\section{Mecanismos de transferencia de calor}
\subsection{Conducción}
	Es la transferencia de energía térmica que se transmite por contacto entre las partículas del material, la razón de conducción es única según el tipo de material e influye su geometría y espesor del mismo. De manera conceptual se puede enunciar que \textit{la razón de la transferencia de calor por conducción a través de una capa plana es proporcional a la diferencia de temperaturas y a la superficie de contacto por donde se transfiere el calor, pero inversamente proporcional a su espesor del material}
	\newline
	Del enunciado anterior se puede representar en la siguiente ecuación agregando la $k$ como la conductividad térmica, que indica la capacidad que tiene un material para poder transferir energía térmica\cite{Cengel2022}.
	\begin{equation}
		\dot{Q_{c}}=kA\frac{\Delta T}{\Delta x}
	\end{equation}
\subsubsection{Difusividad térmica}
	Esta propiedad de los materiales es útil para el análisis de la conducción de calor, debido a que representa la razón de difusión la energía térmica a través del material\cite{Cengel2022}.
	\begin{equation}
		\alpha = \frac{k}{\rho c_p}
	\end{equation}
	Donde $k$ indica la conductividad térmica, y el producto de la densidad con la capacidad calorífica $\rho c_p$ representa la cantidad de energía que se puede contener.
\subsection{Convección}
Esta manera de transferir el calor entre un material sólido y líquido o gaseoso, es una combinación del movimiento de los fluidos y la conducción térmica. Este mecanismo se puede representar como la rapidez de cuanto calor transferido por conducción es proporcional a la diferencia entre las temperaturas del solido con el liquido o gas. Que es conocida como la \textbf{ley de enfriamiento de Newton}\cite{Cengel2022}:
	\begin{equation}
		\dot{Q_{cv}} = hA_s(T_s-T_\infty)
	\end{equation}
Donde el coeficiente de transferencia de energía térmica por convección es representado por $h$, $A_s$ la superficie de contacto donde se esta transfiriendo calor, $T_s$ la temperatura del área de contacto y $T_\infty$ corresponde a la temperatura del liquido o gas que esta lo suficientemente lejano del solido.	
\subsection{Radiación}
 Este mecanismo transfiere energía térmica en forma de ondas electromagnéticas, una característica principal de este método es que no requiere un medio como en el caso de la conducción o convección. Esta forma de transferencia se puede expresar mediante la \textbf{ley de Stefan-Boltzmann}\cite{Cengel2022} para superficies reales:
 \begin{equation}
 	\dot{Q_r}=\epsilon \sigma A_s T_{s}^4
 \end{equation}
 El coeficiente $\sigma$ representa la \textit{constante de Stefan-Boltzmann}, $\epsilon$ corresponde a la emisividad del material que esta en un intervalo $0\leq \epsilon \leq 1$, $A_s$ superficie que emite la radiación y su temperatura termodinámica $T_s$ en Kelvin. 
\section{Almacenamiento de energ\'ia t\'ermica}
En el desarrollo de sistemas de almacenamiento de energ\'ia t\'ermica se deben considerar los siguientes factores:
\begin{itemize}
	\item \textbf{Capacidad}: Esta característica del sistema depende el proceso de almacenamiento, el medio y el tama\~no que tendrá, por ejemplo el agua tiene una capacidad de almacenar calor de $10-50 \frac{kWh}{t}$.La capacidad de almacenamiento tiene un importante efecto en la operación del resto del sistema.
	\item \textbf{Potencia}: Es la raz\'on de carga y descarga relacionados con la potencia necesaria. De manera general, se define que tan r\'apido la energ\'ia puede almacenarse en un sistema y descargarse.
	\item \textbf{Eficiencia}: Esta relaci\'on entre la energ\'ia que suministra el usuario con la energ\'ia que se requiere para cargar el sistema. Esto considera la perdida de energ\'ia durante el almacenamiento en un periodo de carga y descarga. Generalmente la eficiencia del calor sensible en un sistema esta en un rango del 50$\%$ - 90$\%$
	\item \textbf{Periodo de almacenamiento}: La evaluaci\'on económica de un sistema de almacenamiento t\'ermico depende significativamente de la aplicaci\'on, los requerimientos de operaci\'on y la frecuencia de almacenamiento.
	\item \textbf{Costo}: Esto se refiere a la capacidad o potencia del almacenamiento del sistema relacionado con los costos de operaci\'on, equipo de almacenamiento y tiempo de vida \'util. El costo incluye desde el inicio del almacenamiento medio, contenedores y aislamientos. De manera general, los costos de estos sistemas pueden variar entre 0.15-10$\frac{\$}{kWh}$ para sistemas de almacenamiento de calor sensible.
\end{itemize}
Cabe mencionar que una buena unidad de almacenamiento debe tener una eficiencia mayor al 80$\%$. Los requerimientos t\'ecnicos para estos sistemas deben incluir:
\begin{itemize}
	\item Alta densidad energ\'etica en el material del almacenamiento, que se traduce en el costo por el espacio ocupado y su aislamiento.
	\item Buena transferencia de calor entre el fluido de transferencia y el medio de almacenamiento con baja p\'erdida t\'ermica.
	\item Mec\'anica y qu\'imicamente estable material de almacenamiento, entre mejor sea su costo incrementa.
	\item Reversibilidad para gran cantidad ciclos de carga y descarga. El costo del intercambiador para la carga y para la descarga es importante\cite{Yatish2017}.
\end{itemize}

\subsection{Almacenamiento de calor sensible}
Los almacenes de calor sensible, usan la capacidad calor\'ifica y el cambio de la temperatura del material del almacén durante el proceso de carga y descarga. La cantidad de calor almacenado depende de su capacidad calor\'ifica, de su cambio de temperatura y la cantidad de almacenamiento t\'ermico que posee el material de interés\cite{Yatish2017}.
\begin{equation}
	Q=\rho V C_p \Delta T
\end{equation}
Donde: 
\newline
$Q$ es la cantidad de calor almacenado en Joules [$J$] 
\newline
$\rho$ es la densidad del material de almacenamiento [$\frac{kg}{L}$]
\newline
$C_p$ es el calor espec\'ifico sobre el rango de temperatura de operaci\'on [$\frac{J}{kg K}$]
\newline
$V$ es el volumen del material de almacenamiento usado [$L$]
\newline
$\Delta T$ es el rango de la temperatura operaci\'on [$^\circ C$]
\newline
\newline
La conductividad  t\'ermica del material afecta directamente la raz\'on carga y la descarga  en el almacenamiento, y se puede obtener de la siguiente expresi\'on:
\begin{equation}
	\lambda=\rho C_p \alpha
\end{equation}
Donde:
\newline
$\lambda$ es la conductividad t\'ermica [$\frac{W}{m K}$]
\newline
$\rho$ es la densidad [$\frac{kg}{m^3}$]
\newline
$C_p$ es la capacidad calor\'ifica [$\frac{J}{kg K}$]
\newline
$\alpha$ es la difusividad t\'ermica [$\frac{m^2}{s}$]