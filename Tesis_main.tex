% !TEX encoding = UTF-8 Unicode
% !BIB TS-program = biber

% This file is MIT-Thesis.tex, a LaTeX template for formatting an MIT thesis with the mitthesis class.
%
% Version: 1.15, 2024/08/31
%
% Author: John H. Lienhard, copyright 2024. Reuse under the MIT license: https://ctan.org/license/mit 

% Documentation is here: https://ctan.org/pkg/mitthesis

%% Don't modify the \DocumentMetadata command unless you know what it does. 
%% If this command throws an "undefined" error, your latex system is out of date: try commenting this command out.
\DocumentMetadata{ 
	pdfstandard = a-2b,
	pdfversion  = 1.7,
	lang		= es-MX,
%	 pdfversion  = 2.0,
%    pdfstandard = a-4,
}

%%%%%%%%%%%%%%%%%%%%%%%%%%%%%%%%%%%%%%%

\documentclass[twoside]{mitthesis}% fontset=newtx, fontset=libertine, fontset=newtx-sans-text, fontset=heros-stix2, fontset=stix2
%
% option [twoside]		gives facing-page behavior for printing; omitting twoside will eliminate even-numbered blank pages.
% option [lineno]	 	provides line numbers, as for editing
% option [mydesign] 	loads packages for color, title and list formats, margins, or captions: edit mydesign.tex to change defaults.
% option [fontset] is a keyvalue which can be:
%					 	for pdftex or unicode engines:  defaultfonts, libertine, lucida
%					 	for pdftex only: 				fira-newtxsf, newtx, newtx-sans-text
%						for unicode engines (luatex):	heros-stix2, stix2, termes, termes-stix2
%					 	if no key value is given, fonts default to CMR (pdftex) or LMR (unicode), i.e., "the LaTeX font".
%					 	You can edit the fontset files or you can write your own, myfonts.tex, and do [fontset=myfonts].
%						If you are using multiple languages, load the babel package in your fontset file, before the fonts.

%%%%%%%%% Packages used in sample chapters (not otherwise required) %%%%%%%

%% Package for code listing in Appendix A.
\usepackage[spanish]{babel}
\usepackage{listings}%   documentation is here https://ctan.org/pkg/listings

%% Set chemical formulas nicely
\usepackage[version=4]{mhchem}%   documentation at https://ctan.org/pkg/mhchem

%% Latin filler used in Chapter 1, with a test for package version date (https://ctan.org/pkg/lipsum)
\usepackage{lipsum}
\IfPackageAtLeastTF{lipsum}{2021/09/20}{\setlipsum{auto-lang=false}}{}

%% Table related packages  

\usepackage{booktabs}% publication quality tables (https://ctan.org/pkg/booktabs)

\usepackage{array}% Additional options for column formats (https://ctan.org/pkg/array)

\usepackage{dcolumn}% For alignment of numbers on the decimal place (https://ctan.org/pkg/dcolumn) 
	\newcolumntype{d}[1]{D{.}{.}{#1}}% use with dcolumn package
	% syntax: d{x.y} where x is max number of digits before decimal and y is max number after.

% Package for multipage table in Appendix B.
\usepackage{longtable}% typeset multi-page tables (https://ctan.org/pkg/longtable)

%\usepackage{tabularx}% adjustable-width columns in tabular (https://ctan.org/pkg/tabularx)


%% Package for improved typography

\usepackage{microtype}% typographic fine-tuning, used in sample thesis committee page, but also acting globally on the text 


%%%%%%%%%  Graphics path (to figure files)  %%%%%%%%%%%%%%%%%%%%%%%%%%%%%%%%

%% Can set graphicspath to point to specific directories containing figures (the current directory is searched automatically)
%% For instance, to search a subdirectory of the current directory called "figures" and a parallel directory called "art", set:

% \graphicspath{ {figures/} {../art/} }% For details see: https://latexref.xyz/dev/latex2e.html#g_t_005cgraphicspath


%%%%%%%%%  Representative set-up for biblatex  %%%%%%%%%%%%%%%%%%%%%%%%%%%%%

%% Numerical citations of references
\usepackage[style=ext-numeric-comp,giveninits=true,maxbibnames=10,sorting=none]{biblatex}

%% IEEE style citations and references
% \usepackage[style=ieee,maxbibnames=10,sorting=none]{biblatex}% style=ext-numeric-comp,articlein=false,giveninits=true
%	 \DefineBibliographyStrings{english}{url= \textsc{url} ,  }% replaces the IEEE default "[Online]. Available" by "URL"

%% author-year style citations and references 
%% use \parencite, not \cite, when you want "(Author, year)"
%% The sample files are not set up to include parentheses.
% \usepackage[style=authoryear, maxbibnames=10]{biblatex} 


\addbibresource{mitthesis-sample.bib}%% <== change to YOUR bib file <= CHANGE

%% to avoid split urls and stretched white space, you can set the bibliography ragged-right:
%\appto{\bibsetup}{\raggedright}

%% biblatex is very powerful, and you can customize most aspects the reference list and citations to suit your needs.
%%   documentation is here: https://ctan.org/pkg/biblatex
%%   cheat sheet is here:   https://tug.ctan.org/info/biblatex-cheatsheet/biblatex-cheatsheet.pdf

%% To ensure citations are set, run Latex --> biblatex/biber --> Latex again

%%%%%%%%%%  Option to use natbib   %%%%%%%%%%%%%%%%%%%%%%%%%%%%%%%%%%%%%%%%%

%\RequirePackage[numbers,sort&compress]{natbib}
 
%%% add bibliography to table of contents
%\apptocmd{\bibliography}{\addcontentsline{toc}{chapter}{\protect\textbf{\bibname}}}{}{}

%%% You can use this to rename the bibliography section
%\renewcommand{\bibname}{References}

%%% To adjust space between bibliography items 
%\setlength\bibsep{4pt plus 1pt minus 1pt}
%   change 4pt to something else; don't drop last two lengths (they are stretchable "glue")


%%%%%%%%%%  Option for "double spacing" %%%%%%%%%%%%%%%%%%%%%%%%%%%%%%%%%%%%

%% Back in the typewriter era, double spaced lines were convenient for editing with a pencil. 
%% In typography, the separation between lines is called "leading", and it is usually set in 
%% proportion to the font size (i.e., when the font is loaded).  If you really feel the need 
%% to change the line separation, the most attractive results will be obtained by changing the
%% leading in proportion to the the current font size, rather than just doubling the space.

%% The setspace package provides a tool for changing line separation. Use these two commands here:
%
% \usepackage{setspace}%  documentation at https://ctan.org/pkg/setspace
% \setstretch{1.1}% you can choose some other value for the stretch of space between lines
%
%% Use one or more of the these commands *AFTER* the frontmatter
%
% \onehalfspacing
% \doublespacing
% \singlespacing  % will turn these effects off (you can use these anywhere in the document)

%% The best result is usually to stay with leading selected by the typographer who set up the font.


%%%%%%%%%%%  Metadata  %%%%%%%%%%%%%%%%%%%%%%%%%%%%%%%%%%%%%%%%%%%%%%%%%%%%%%%

% Most of the document metadata is created automatically. 
% The following items should be adjusted to match your work. <================= !!!!!!!!!!

\hypersetup{%
	pdfsubject={Plantilla modificada del Massachusetts Institute of Technology},
	% Change this to briefly state topic of your thesis 
% 
	pdfkeywords={Centro Nacional de Investigación y Desarrollo Tecnológico, Cenidet},
	% Add keywords that will help search engines and libraries to find your work.
	% Includes the name[s] of the author[s] 
	% (If you used \DocumentMetadata at line 14, you can just put "\CopyrightAuthor," for the names.)
%
	pdfurl={},
	% If you have a url for the thesis, put it here. Otherwise delete this.
	% (MIT Libraries will put your thesis in DSPACE with a persistent url after you submit it.)
%	
	pdfcontactemail={},
	% You can put a [permanent] email address into the metadata, if you like.
	% Otherwise delete this.
%
	pdfauthortitle={},
	% If you have a title, you can include it here.
}

%%%%%%%%%%%%%%  End preamble %%%%%%%%%%%%%%%%%%%%%%%%%%%%%%%%%%%%%%%%%%%%%%%%%%%%%%%%%%%%%%%%%%%%%
%%%%%%%%%%%%%%%%%%%%%%%%%%%%%%%%%%%%%%%%%%%%%%%%%%%%%%%%%%%%%%%%%%%%%%%%%%%%%%%%%%%%%%%%%%%%%%%%%%

\begin{document}

%%% edit the following commands to match your thesis %%%%%%%%%%

\title{Diseño, construcción y puesta en marcha de un regenerador de energía para el desarrollo y validación de estrategias de modelado matemático}

% \Author{Author full name}{Author department}[Author's first PREVIOUS degree][Author's second PREVIOUS degree][...
% Note that third, fourth, fifth, and sixth arguments are optional [] and may be omitted

% note on names: most of the following names are made up; Silas Holman was a physics professor at MIT in the 19th century.

\Author{Omar Arturo Castillo Méndez}{Departamento de Ingenieria Electrónica}
% \Author{Luisa Hernández}{Department of Research}[B.S. Mechanical Engineering, UCLA, 2018][M.S. Stellar Interiors, Vulcan Science Academy, 2020]
% \Author{Thurston Howell III}{Department of Economics}[MBA, Ferengi School of Management, 2022]

% Use once for each degree fulfilled by thesis
% For two degrees from one department, leave the department argument blank for the second degree {}.
\Degree{Doctor en Ciencias en Ingeniería Electrónica}{Departamento de Ingenieria Electrónica}
%\Degree{Master of Science in Physics}{}
%\Degree{Bachelor of Science in Mechanical Engineering}{Department of Mechanical Engineering}

% Si tienes director(a) //codirector(a) de tesis.
\Supervisor{Victor Manuel Alvarado Martinez}{Doctor en Ciencias}
\Supervisor{Ma. Guadalupe López López}{Doctor en Ciencias}
% \Supervisor{Edward C. Pickering}{Professor of Physics, and \\ \> Professor of Something Else}
% \Supervisor{Secunda Castor}{Professor of Research}
% \Supervisor{Quintus Castor}{Professor of Log Dams}

% Professor who formally accepts theses for your department (e.g., the Graduate Officer, Professor Sméagol,...)
% If more than one department, use more than once
\Acceptor{José Francisco Gómez Aguilar}{Doctor en Física}{Cenidet}
\Acceptor{Ricardo Fabricio Escobar Jiménez}{Doctor en Ciencias}{Cenidet}
\Acceptor{Jarniel García Morales}{Doctor en Ciencias}{Cenidet} 
% \\ \> Third title}
% \Acceptor{Quarta Castor}{Professor of Lodge Building}{Graduate Officer, Department of Mechanical Engineering}
%%%  If you need to reduce vertical space, put the acceptor title in the second argument and leave the third blank, {}.
% \Acceptor{Primus Castor}{Professor and Undergraduate Officer, Department of Physics}{}

% Usage: \DegreeDate{Month}{year} Fecha de obtencion de grado
% Usar fechas validas Enero, Febrero, Marzo, Abril, Mayo, ..., Diciembre
\DegreeDate{Septiembre}{2026}

% Fecha de la tesis autorizacion de entrega al departamento
\ThesisDate{Junio, 2026}


%%%%%%  Choose whether to have a CREATIVE COMMONS License  %%%%%%%%%%%%%%%%%%%%%%%%%%%%%%%%%%%%%%
%
% If you are using a cc license, put details of your cc license here. 
% Omit this command if you are not using a cc license.
%
\CClicense{CC BY-NC-ND 4.0}{https://creativecommons.org/licenses/by-nc-nd/4.0/}
%

%%%%%%%  Solutions for overflowing titlepage  %%%%%%%%%%%%%%%%%%%%%%%%%%%%%%%%%%%%%%%%%%%%%%%%%%%

% If your title page is overflowing (from too many names, degrees, etc.):
%
% (a) you can reduce the 12pt and 18pt skips between various blocks to 6pt with this command:
%
% \Tighten
%
% (b)  you can scale down the Signature block at the bottom with this command:
%
% \SignatureBlockSize{\small}  %or this one \SignatureBlockSize{\footnotesize}
%
% (c) you can put the acceptor name and title onto two lines, rather than three like this:
%
% \Acceptor{Tertius Castor}{Professor and Graduate Officer, Department of Research}{}
%
% (d) you can change the font size of the author name[s] with
%
%	\AuthorNameSize{\normalsize}
%
% (e) and you can omit any previous degrees from the title page, instead mentioning them in the biographical sketch

% Also, if you prefer to keep the text toward the top of the page with most white space at the bottom, you
% can use this command to squash all of the vertical glue (stretchy space) with this command:
%
% \Squash 
%
% This command is useful when the text has not already reach the bottom of the page, since the glue gets squashed automatically
% when the page is too full.

%%%%%%%%%%%%%%%%%%%%%%%%%%%%%%%%%%%%%%%%%%%%%%%%%%%%%%%%%%%%%%%%%%%%%%%%%%%%%%%%%%%%%%%%%%%%%%%%%

%%% Make titlepage
\maketitle

%%%%%%%%% Contents that you need to write follows! %%%%%%%%%%%%%%%%%%%%%%%%%%%%%%%%%%%%%%%%%%%%%%

% \includeonly{acknowledgments,biography,chapter1,chapter2,...,appendixa,...} 
%   for usage of includeonly, see https://latexref.xyz/_005cinclude-_0026-_005cincludeonly.html

%%% Frontmatter (write this material in the mentioned files)  %%%%%%%%%%%%%%%%%%%%%%%%%%%%%%%%%%%

% This page is optional. Edit the file committee_members.tex 
% Sample thesis committee page for mitthesis.cls
% Version 1.00, 2024/08/24
%
% This page is not required by the MIT Libraries, but some departments require it
%
% Insert between title page and abstract page.
% Format this page in any way that you like.  
% Add supervisor titles, degrees, and departments as appropriate

%%%%% FORMATTING COMMANDS %%%%%%%%%%%%%%%%%%

%% Format title
\NewDocumentCommand\CommitteePageTitle{m}{
	\vspace*{75pt}%36pt}
	\IfPackageLoadedTF{microtype}
		{\textls*{\Large\textbf{\MakeUppercase{#1}}}}
		{{\Large\textbf{\MakeUppercase{#1}}}}%
	\pdfbookmark[0]{#1}{Committee}%
	\vspace*{10pt}%
}
% \textls* produces additional letter separation (appropriate for capitalized display text),
% PROVIDED THAT \usepackage{microtype} has been loaded in the preamble. 
% The extra space added is 100/1000 em (adjustable, see package documentation).

%% Format committee member subheadings
\NewDocumentCommand\Role{m}{
	\vspace*{50pt}%25pt}
	\IfPackageLoadedTF{microtype}
		{\textls*{\large{\textsc{#1}}}}
		{{\large\textsc{#1}}}%
	\vspace*{12pt}%
}
%%%%%%%%%%%%%%%%%%%%%%%%%%%%%%%%%%%%%%%%%%%%%

\begin{flushright}

\CommitteePageTitle{Comité de tesis}

\Role{Director de Tesis}

 \textbf{Victor Manuel Alvarado Martínez} \\
 {\itshape
 Doctor en Ciencias \\
 Departamento de Ingeniería Electrónica \\
 }

\Role{Codirectora de Tesis}

\textbf{Ma. Guadalupe López López} \\
{\itshape
	Doctora en Ciencias \\
	Departamento de Ingeniería Electrónica \\
}


\Role{Revisores de tesis}

 \textbf{José Francisco Gómez Aguilar} \\
 {\itshape
   Doctor en Física \\
   Departamento de Ingeniería Electrónica \\[18pt]
 }

 \textbf{Ricardo Fabricio Escobar Jiménez}\\
 {\itshape
   Doctor en Ciencias \\
   Departamento de Ingeniería Electrónica \\[18pt]
 }

 \textbf{Jarniel García Morales} \\
 {\itshape
   Doctor en Ciencias \\
   Departamento de Ingeniería Electrónica \\
 }

\end{flushright}

\cleardoublepage


% The abstract environment creates all the required headings and footers. 
% You only need to the text of the abstract in the file abstract.tex
\begin{abstract}
	% From mitthesis package
% Version: 1.01, 2023/06/19
% Documentation: https://ctan.org/pkg/mitthesis
%
% The abstract environment creates all the required headers and footnote. 
% You only need to add the text of the abstract itself.
%
% Approximately 500 words or less; try not to use formulas or special characters
% If you don't want an initial indentation, do \noindent at the start of the abstract

Aqui se escribe un abstract de la tesis% use \input rather than \include because we're inside an environment
\end{abstract}

%% acknowledgments.tex

% From mitthesis package
% Version: 1.02, 2024/06/19
% Documentation: https://ctan.org/pkg/mitthesis

\chapter*{Agradecimientos}
\pdfbookmark[0]{Agradecimientos}{agradecimientos}


Escribe los agradecimientos aquí
% acknowledgments.tex (.tex extension is presumed by \include) 

\include{biography}% biography.tex (optional, see https://libraries.mit.edu/distinctive-collections/thesis-specs/#format)

%%% Table of contents and lists of stuff (delete unused lists, i.e., if no tables or figures) %%%%%

\tableofcontents
\listoffigures
\listoftables

%%% Chapters of thesis  %%%%%%%%%%%%%%%%%%%%%%%%%%%%%%%%%%%%%%%%%%%%%%%%%%%%%%%%%%%%%%%%%%%%%%%%%%%

%% If you want to use "double spacing", you should start here...

 
\chapter{Introducción}

En la actualidad el uso de combustibles f\'osiles ha sido primordial para el desarrollo de la sociedad. En M\'exico, según datos de la Secretaria de Energ\'ia, el consumo total de energ\'ia nacional de gas natural corresponde al 49.11\% el cual mayormente es usado para las termoeléctricas de ciclo combinado y combusti\'on interna\cite{BNE2021}. Para reducir la contaminaci\'on y recuperar calor residual de cualquier proceso debido a la combustión de combustibles fósiles es de vital importancia el estudio de recuperadores de energ\'ia t\'ermica o regeneradores. En un proceso industrial t\'ermico la temperatura residual esta en un rango de $100-200 ^\cdot C$\cite{Anish2021}. 
\newline
Las aplicaciones de los intercambiadores de calor que funcionan como regeneradores de energ\'ia t\'ermica se han investigado de manera puntual según su aplicaci\'on y uso. Se pueden encontrar en la literatura, calentadores de aire solar que son muy costosos para su fabricaci\'on dependiendo del material de construcci\'on, por lo que se buscan técnicas para reducir sus costes de fabricaci\'on y que sean mas accesibles%\cite{Abbas2022}.
\newline
Estos sistemas pueden ser dise\~nados a partir de correlaciones como por ejemplo el uso del n\'umero de Nusselt, esta cantidad adimensional indica como es la transferencia de calor por convecci\'on%\cite{Pimsarn2022}, ademas el numero de Raynolds es de vital importancia para saber el r\'egimen del fluido que se esta estudiando es esta en la categoría de laminar o turbulento %\cite{Kong2022}.
\newline
Otra manera de abordar problemas t\'ermicos es mediante las ecuaciones gobernantes de conservaci\'on de masa%\cite{Zhao2023}, momentum y energ\'ia. A diferencia de la t\'ecnica por correlaciones, las ecuaciones gobernantes se pueden expresar en tres dimensiones ($x, y,$ y $z$)%\cite{Rehman2022}.
Con la ayuda de la tecnol\'ogia, se pueden simular estos sistemas para comprobar los modelos obtenidos y asi tener una referencia mas fiable. La din\'amica de fluidos computacional, es una gran herramienta que ademas de proveer informaci\'on del comportamiento del flujo en la cavidades, principalmente para los que son en medios porosos%\cite{Prajapati2022}.
\newline
Recientemente el modelado computacional y dise\~no artifical de medios porosos ha aumentado en los ultimos a\~nos debido a su aplicaci\'on en las \'areas como la biom\'edica y el\'ectrica. En la investicaci\'on de esos sistemas se enfocan en como es la distribución de los poros o cavidades, y como su geometría afecta en la forma en que se transfiere calor o en la que mejor se aprovecha%\cite{Prajapati2022}.
\newline

\subsection{Objetivo general}
Diseñar, construir, y poner en operaci\'on una planta piloto de un regenerador de energ\'ia de lecho empacado, que sirva como estaci\'on de prueba para validar estructuras matem\'aticas que representen la din\'amica de la
planta y que sirvan para la soluci\'on de problemas de diseño, optimizaci\'on y control de estos sistemas.
\subsection{Objetivos espec\'ificos}
1.- Obtener un prototipo de regenerador de energ\'ia con doble lecho empacado con un monolito met\'alico.
\newline 2.- Generar tres modelos matem\'aticos mediante diferentes m\'etodos de modelado para reproducir el
comportamiento t\'ermico de un periodo de calentamiento o enfriamiento, todos validados con datos
provenientes del prototipo experimental.
\newline 3.- Formular un modelo matem\'atico del comportamiento t\'ermico de un ciclo completo, en r\'egimen pseudo-estacionario.
\subsection{Hip\'otesis}
%\begin{itemize}
%	\item El sistema completo tiene tiempo de enfriamiento y calentamiento de ambos regeneradores es simultaneo cuando hace el cambio a contraflujo.
%	\item El aire usado para el calentamiento o enfriamiento es seco.
%	\item El sistema de suministro de aire solo funciona en las condiciones y lugar del CENIDET
%	\item Las ecuaciones dise\~{n}o del intercambiador de calor van a servir de punto de partida para probar otras t\'ecnicas de modelado.
%\end{itemize}

% .tex extension is presumed
% 
\chapter{Marco conceptual}

\section{Almacenamiento de energ\'ia t\'ermica}
En el desarrollo de sistemas de almacenamiento de energ\'ia t\'ermica se deben considerar los siguientes factores:
\begin{itemize}
	\item \textbf{Capacidad}: Esta característica del sistema depende el proceso de almacenamiento, el medio y el tama\~no que tendrá, por ejemplo el agua tiene una capacidad de almacenar calor de $10-50 \frac{kWh}{t}$.La capacidad de almacenamiento tiene un importante efecto en la operación del resto del sistema.
	\item \textbf{Potencia}: Es la raz\'on de carga y descarga relacionados con la potencia necesaria. De manera general, se define que tan r\'apido la energ\'ia puede almacenarse en un sistema y descargarse.
	\item \textbf{Eficiencia}: Esta relaci\'on entre la energ\'ia que suministra el usuario con la energ\'ia que se requiere para cargar el sistema. Esto considera la perdida de energ\'ia durante el almacenamiento en un periodo de carga y descarga. Generalmente la eficiencia del calor sensible en un sistema esta en un rango del 50$\%$ - 90$\%$
	\item \textbf{Periodo de almacenamiento}: La evaluaci\'on económica de un sistema de almacenamiento t\'ermico depende significativamente de la aplicaci\'on, los requerimientos de operaci\'on y la frecuencia de almacenamiento.
	\item \textbf{Costo}: Esto se refiere a la capacidad o potencia del almacenamiento del sistema relacionado con los costos de operaci\'on, equipo de almacenamiento y tiempo de vida \'util. El costo incluye desde el inicio del almacenamiento medio, contenedores y aislamientos. De manera general, los costos de estos sistemas pueden variar entre 0.15-10$\frac{\$}{kWh}$ para sistemas de almacenamiento de calor sensible.
\end{itemize}
Cabe mencionar que una buena unidad de almacenamiento debe tener una eficiencia mayor al 80$\%$. Los requerimientos t\'ecnicos para estos sistemas deben incluir:
\begin{itemize}
	\item Alta densidad energ\'etica en el material del almacenamiento, que se traduce en el costo por el espacio ocupado y su aislamiento.
	\item Buena transferencia de calor entre el fluido de transferencia y el medio de almacenamiento con baja p\'erdida t\'ermica.
	\item Mec\'anica y qu\'imicamente estable material de almacenamiento, entre mejor sea su costo incrementa.
	\item Reversibilidad para gran cantidad ciclos de carga y descarga. El costo del intercambiador para la carga y para la descarga es importante\cite{Yatish2017}.
\end{itemize}

\subsection{Almacenamiento de calor sensible}
Los almacenes de calor sensible, usan la capacidad calor\'ifica y el cambio de la temperatura del material del almacén durante el proceso de carga y descarga. La cantidad de calor almacenado depende de su capacidad calor\'ifica, de su cambio de temperatura y la cantidad de almacenamiento t\'ermico que posee el material de interés\cite{Yatish2017}.
\begin{equation}
	Q=\rho V C_p \Delta T
\end{equation}
Donde: 
\newline
$Q$ es la cantidad de calor almacenado en Joules [$J$] 
\newline
$\rho$ es la densidad del material de almacenamiento [$\frac{kg}{L}$]
\newline
$C_p$ es el calor espec\'ifico sobre el rango de temperatura de operaci\'on [$\frac{J}{kg K}$]
\newline
$V$ es el volumen del material de almacenamiento usado [$L$]
\newline
$\Delta T$ es el rango de la temperatura operaci\'on [$^\circ C$]
\newline
\newline
La conductividad  t\'ermica del material afecta directamente la raz\'on carga y la descarga  en el almacenamiento, y se puede obtener de la siguiente expresi\'on:
\begin{equation}
	\lambda=\rho C_p \alpha
\end{equation}
Donde:
\newline
$\lambda$ es la conductividad t\'ermica [$\frac{W}{m K}$]
\newline
$\rho$ es la densidad [$\frac{kg}{m^3}$]
\newline
$C_p$ es la capacidad calor\'ifica [$\frac{J}{kg K}$]
\newline
$\alpha$ es la difusividad t\'ermica [$\frac{m^2}{s}$]
%\include{chapter3}
%\include{chapter4}


%%% Appendicies of thesis  %%%%%%%%%%%%%%%%%%%%%%%%%%%%%%%%%%%%%%%%%%%%%%%%%%%%%%%%%%%%%%%%%%%%%%%%

\appendix
\include{appendixa}
\include{appendixb}

%%% Bibliography (biblatex)  %%%%%%%%%%%%%%%%%%%%%%%%%%%%%%%%%%%%%%%%%%%%%%%%%%%%%%%%%%%%%%%%%%%%%%

\defbibheading{bibintoc}{\chapter*{#1}\addcontentsline{toc}{backmatter}{\refname}} 
% this sets the title of contents name for bibliography to \refname (= References)
% change "backmatter" to "chapter" if you prefer a bold face entry in the table of contents

\printbibliography[title={\refname},heading=bibintoc]

% biblatex also supports chapter-by-chapter bibliography, https://tex.stackexchange.com/a/296502/119566
% see the biblatex manual, section 3.14.3


%%%% Option for natbib %%%%%%%%%%%%%

%%   use an appropriate style (.bst) and your own .bib file[s]

%\bibliographystyle{plainnat}
%\bibliography{mitthesis-sample.bib}

\end{document} 
 